\documentclass[12pt,a4paper]{article}

% Preamble
\usepackage{geometry} % to adjust page dimensions
\geometry{a4paper, margin=2in}
\usepackage{hyperref}      % for clickable references
\usepackage{pgfplots}     % for plots
\usepackage{tikz}         % for illustrations
\usepackage{listings}     % for source code
\usepackage[newfloat]{minted}
\usepackage{xcolor}       % for custom colors
\usepackage{booktabs}     % for tables
\usepackage{siunitx}      % for numerical data and units
\usepackage{algorithm2e}  % for pseudo code
\usepackage[utf8]{inputenc} % for special characters
\usepackage{amsmath}      % for enhanced math environments
\usepackage{tabularx}  % for tabularx environment

\usemintedstyle{tango}

\title{Assignment 1\\Sorting in Assembly using Bubble Sort}
\author{Max Neuwinger, Yerim Hong}
\date{}

\begin{document}
\maketitle

\section{Introduction}
This project involves the implementation of a program in x86-64 Linux assembly language. The program's primary task is to read (x, y) coordinates from a given file, sort them based on their y-values, and output the sorted coordinates. This report provides a comprehensive explanation of the code implementation while evaluating its performance.

\section{Design Decisions}
Bubble Sort was selected due to its straightforward logic and inherent in-place sorting nature. The mechanism involves iterating over the list multiple times, comparing neighboring elements, and making swaps when necessary until the entire list becomes ordered. Given the scope of this project—working with (x, y) coordinates and ensuring a clear demonstration of assembly language programming—the ease of Bubble Sort's implementation made it an appropriate choice. Although it may not be the most efficient for extensive datasets, its implementation in assembly language serves as an effective illustration of sorting algorithms in a low-level language setting.

\section{Implementation}
The process began by configuring an array to store the (x, y) coordinates read from the input file. Each element of the array represented a coordinate pair. Proper initialization and memory allocation were guaranteed. Following this setup, the Bubble Sort algorithm was employed to sort these coordinates based on their y-values. The sorted (x, y) pairs were then formatted and outputted in the same format as the input file.

\subsection{Bubble Sort Implementation}

The Bubble Sort implementation aims to sort the (x, y) coordinates based on their y-values. The sorting process takes advantage of registers for optimized handling of data during the sorting mechanism.

\subsubsection{Initialization}

Initially, the starting address of the parsed data is backed up in the \texttt{\%r8} register. The number of (x, y) pairs is loaded into the \texttt{\%r9} register. By shifting left by 4, it's converted to bytes, since each pair is 16 bytes: 8 bytes for x and 8 for y. The \texttt{\%r9} register is then updated to point to the end of the list.

\subsubsection{Outer Loop}

The outer loop is responsible for going through the entire array. The start address for this loop is stored in the \texttt{\%r10} register, and the swap flag is initialized to 0 using \texttt{\%r15} register.

\subsubsection{Inner Loop and Swapping}

\begin{minted}{gas}
inner_loop:
    ...
    movq 8(%r10), %r11         
    cmpq -8(%r10), %r11        
    jge no_swap                

    ...
    movq $1, %r15              
    jmp inner_loop             
\end{minted}

The inner loop performs comparisons between adjacent (x, y) pairs based on their y-values. If the y-value of the current pair is smaller than the y-value of the next pair, the pairs are swapped. The swap operation involves saving the values of the pairs in temporary registers and then interchanging their positions in the array. Once a swap occurs, the swap flag is set to 1.

\subsubsection{Checking and Finalizing}

\begin{minted}{gas}
check_swap:
    testq %r15, %r15           
    jz sorting_done            

    subq $16, %r9              
    cmpq %r9, %r8              
    je sorting_done
    jmp outer_loop             
\end{minted}

After completing the inner loop, the program checks if any swaps were made using the swap flag. If no swaps occurred, the sorting process is complete. If swaps did happen, the pointer indicating the end of the current scope for sorting is decremented, and the outer loop is executed again. This narrowing of scope ensures optimization in the number of comparisons made during the sort.

\subsubsection{Termination}

Upon ensuring that the data is fully sorted, the program exits the sorting routine.

\subsection{Print}

Following the sorting of the coordinates, the program proceeded with the printing function. This function traversed the sorted data array, with the \texttt{r8} register serving as a pointer and the \texttt{r9} register managing the loop count.

To print the (x, y) coordinate pairs, the program employed two distinct procedures: printX and printY. The printX procedure was responsible for displaying the x-coordinate, followed by a tab character. Conversely, the printY procedure showcased the y-coordinate, ending with a newline character. Each coordinate was loaded into the \texttt{rdi} register before the respective print function was invoked, ensuring accurate representation and formatting during the printing process.

The program continued this process until all coordinate pairs from the sorted array were printed.

\subsection{Error Handling}
Various error-handling routines have been incorporated in the program to address potential issues that might arise during its execution. The following are the identified error conditions:

\begin{itemize}
    \item \textbf{No Filename Argument}: Triggered when no filename is provided as a command-line argument. 
    \item \textbf{Too Many Arguments}: Activated when multiple arguments are passed, which the program doesn't expect.
    \item \textbf{Open Error}: Occurs when the program encounters issues opening a specified file.
    \item \textbf{Read Error}: Arises when the program fails to read the content of the file.
    \item \textbf{Partial Read Error}: This error is encountered when there's a problem during file reading, possibly due to unexpected EOF or file corruption.
    \item \textbf{No Lines in File}: Triggered when the file being processed contains no lines.
\end{itemize}

All these error handlers invoke the \texttt{handle\_error} routine to display the corresponding error message and subsequently exit the program.

\section{Program Evaluation}

This section systematically presents the performance evaluation of the sorting program, with datasets of varying sizes, and offers a comprehensive discussion of the results, comparing the observed findings with theoretical expectations.

\subsection{Dataset and Procedure}

Datasets of coordinates, ranging from 10,000 to potentially 5,000,000, were employed for the assessment. To ensure robustness, each dataset size was subjected to three trials.

\textbf{Benchmarking Steps:}
\begin{enumerate}
    \item Generate uniformly random coordinates across multiple instances for each dataset size.
    \item Record the program's “real”/“wall” time using the \texttt{time} command.
    \item Count the number of comparison operations during each sorting process.
    \item Derive the rate of comparisons per second for each test instance.
\end{enumerate}

\subsection{Runtime Evaluation and Discussion}

\begin{table}[ht]
    \centering
    \small
    \begin{tabular}{ccc}
        \toprule
        number of coordinates & time in seconds & cmp Instructions per second \\
        \midrule
        10,000      & 0.125           & 799,716,480 \\
        50,000      & 4.501           & 555,427,535 \\
        100,000     & 17.997          & 555,556,611 \\
        500,000     & 440.835         & 567,137,667 \\
        1,000,000   & 1851.317        & 540,203,207 \\
        \bottomrule
    \end{tabular}
    \caption{Runtimes and cmp Instructions per second of the sorting program.}
\end{table}

The data reveals a notable trend: smaller datasets have a comparatively higher rate of `cmp` instructions per second. Several factors underpin this observation:

\begin{itemize}
    \item \textbf{Startup Overhead}: Such overheads have a more pronounced effect on smaller datasets, elevating their `cmp` rates.
    \item \textbf{Memory Hierarchy}: Smaller datasets benefit from efficient cache utilization, which expedites comparisons.
    \item \textbf{Algorithmic Behavior}: While Bubble Sort tends to have a quadratic behavior, certain inputs or early termination conditions might influence its performance.
\end{itemize}

\subsection{Theoretical Complexity Evaluation and Discussion}

\begin{table}[ht]
    \centering
    \small
    \begin{tabular}{ccc}
        \toprule
        number of coordinates & theoretical complexity & cmp Instructions \\
        \midrule
        10,000      & 100,000,000            & 99,964,560 \\
        50,000      & 2,500,000,000          & 2,499,904,418 \\
        100,000     & 10,000,000,000         & 9,999,618,570 \\
        500,000     & 250,000,000,000        & 249,996,989,360 \\
        1,000,000   & 1,000,000,000,000      & 999,996,505,180 \\
        \bottomrule
    \end{tabular}
    \caption{Comparison of theoretical complexity and observed cmp Instructions.}
\end{table}

The table above compares the theoretical complexity of Bubble Sort (\(O(n^2)\)) with the observed number of comparisons. It's evident that the actual program performance closely mirrors the theoretical expectations, even if there are minor deviations due to machine-specific or input-specific conditions. 

Such alignment between theory and practice underscores the reliability of the implemented sorting program. It also suggests that while Bubble Sort may not be the most efficient for larger datasets, the consistency in its behavior can be leveraged for datasets within the observed size range.


\subsection{Overhead time Evaluation and Discussion}

\begin{table}[ht]
    \centering
    \small
    \begin{tabular}{ccc}
        \toprule
        number of coordinates & time in seconds \\
        \midrule
        10,000      & 0.036 \\
        50,000      & 0.180 \\
        100,000     & 0.364 \\
        500,000     & 1.839 \\
        1,000,000   & 3.649 \\
        \bottomrule
    \end{tabular}
    \caption{Running time excluding the sorting time.}
\end{table}

The table above illustrates the correlation between the number of coordinates and the corresponding overhead time, excluding the sorting time. As the number of coordinates in the file increases, the overhead time exhibits a linear growth pattern, aligning with the theoretical expectation of \( O(n) \) complexity. This linear relationship validates the theoretical analysis, demonstrating that the program's overhead time scales linearly with the file size, a characteristic expected in \( O(n) \) algorithms.



\section{Discussion}
The implementation of the Bubble Sort algorithm in assembly language provided an opportunity to examine algorithm performance at a granular level. Several observations and insights emerged from this investigation:

\begin{itemize}
    \item \textbf{Performance Discrepancies}: While the program evaluation revealed a stark increase in runtime for larger datasets, this is consistent with Bubble Sort's \(O(n^2)\) time complexity. Despite this expected behavior, smaller datasets were sorted quite efficiently, hinting that the algorithm's simplicity can be advantageous for certain applications or size constraints.
    
    \item \textbf{Assembly's Intimate Interaction}: Programming in assembly allowed for a detailed observation of the program's operations. For example, the direct tracking of comparison instructions gave a precise measure of the algorithm's inner workings and reinforced the theoretical complexities of Bubble Sort.
    
    \item \textbf{Cache Utilization}: The varying rates of `cmp` instructions per second, especially with smaller datasets, highlight the importance of hardware-level intricacies like cache memory. When datasets fit well within cache sizes, retrieval times decrease, leading to faster comparisons and an overall boost in performance.
\end{itemize}

The results underscore the significance of algorithmic choices, especially in low-level languages like assembly. While more efficient algorithms might be chosen for larger datasets, understanding the foundational principles and performance characteristics of simpler algorithms like Bubble Sort remains crucial.

\section{Conclusion}
In this project, we explored low-level programming and algorithm design using Bubble Sort as a case study. Its behavior in assembly, along with performance insights, underscores its importance as a basic sorting method.
% If you have references
% \bibliographystyle{plain}
% \bibliography{report}

\end{document}
